\pagebreak
\section{Aufgabe 2: Neuronale Netze}
\subsection{Unterkapitel 1}
\subsubsection{Unterkapitel 1.1}
\subsection{Neuronale Netze mit TensorFlow}
%TODO: Tobi
Im Folgenden wird die Implementierung eines voll vernetzen Neuronalen Netzwerks mit Hilfe der \emph{TensorFlow} Bibliothek beschrieben \cite{tensorflow2015-whitepaper}. Unterstützend wurde für den Aufbau des neuronalen Netzes die \emph{Keras}-API verwendet, die seit \emph{TensorFlow} Version 2 standardmäßig integriert ist. 

Um die Daten für das Training des Neuronalen Netzes effizient vorzubereiten wurde die \emph{Dataset}-API verwendet. Mit Hilfe der \emph{Dataset}-API wurde das zufällige Mischen der Trainingsdaten, sowie die Bereitstellung in Form von \emph{Mini-Batches} implementiert. Die Verwendung der \emph{Dataset}-API hat noch zusätzlich den Vorteil, dass die Ausführung von Vorbereitungsschritten der Daten perfekt mit dem Training des Neuronalen Netzes koordiniert und parallelisiert werden können.

\subsubsection{Hyperparameter Suche}
Um ein best Mögliches Modell für die vorliegenden Daten zu finden, wurde eine Hyperparameter-Suche implementiert.
%TODO: Details und halt implementieren.
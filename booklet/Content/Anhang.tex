\pagebreak
\pagenumbering{Roman}
\section{Anhang}
\subsection{Ergänzende Abbildungen zu Booklet Teil 1}
\label{app:abb_booklet_1}
Folgende Abbildungen stellen die Entscheidungsbäume der Cost-Complexity Aufgabe dar. Der Unterschied in den Bäumen mit geringerem und höherem $\alpha$ wird für die beiden Baum-Variationen gut ersichtlich. Abbildungen \ref{fig:ccp_fulltree_alpha_0005} und \ref{fig:ccp_fulltree_alpha_002} zeigen je einen vollständig ausgebildeten Baum, der durch Cost-Complexity \gqq{beschnitten} wurden.\\
\begin{figure}[H]
    \centering
     \begin{minipage}{0.45\textwidth}
        \centering
        \includegraphics[width=0.9\textwidth]{Bilder/ccp_fulltree_alpha_0005.png}
        \caption{Entscheidungsbaum mit Parametern \emph{max\_depth=None} und \emph{ccp\_alpha=0.0005}}
        \label{fig:ccp_fulltree_alpha_0005}
    \end{minipage}\hfill
    \begin{minipage}{0.45\textwidth}
        \centering
        \includegraphics[width=0.9\textwidth]{Bilder/ccp_fulltree_alpha_002.png}
        \caption{Entscheidungsbaum mit Parametern \emph{max\_depth=None} und \emph{ccp\_alpha=0.002}}
        \label{fig:ccp_fulltree_alpha_002}
    \end{minipage}\hfill
\end{figure}
Abbildungen \ref{fig:ccp_maxDepth_alpha_001} und \ref{fig:ccp_maxDepth_alpha_004} zeigen je einen beschnittenen Baum der mit der Einstellung \emph{max\_depth = 8} erstellt wurde.
\begin{figure}[H]
    \centering
     \begin{minipage}{0.45\textwidth}
        \centering
        \includegraphics[width=0.9\textwidth]{Bilder/ccp_maxDepth_alpha_001.png}
        \caption{Entscheidungsbaum mit Parametern \emph{max\_depth=8} und \emph{ccp\_alpha=0.001}}
        \label{fig:ccp_maxDepth_alpha_001}
    \end{minipage}\hfill
    \begin{minipage}{0.45\textwidth}
        \centering
        \includegraphics[width=0.9\textwidth]{Bilder/ccp_maxdepth_alpha_004.png}
        \caption{Entscheidungsbaum mit Parametern \emph{max\_depth=8} und \emph{ccp\_alpha=0.004}}
        \label{fig:ccp_maxDepth_alpha_004}
    \end{minipage}\hfill
\end{figure}
\pagebreak
\subsection{Quellcode zu Booklet Teil 1}
\pagebreak
\subsection{Ergänzungen zu Booklet Teil 2} \label{app:ergaenzung_booklet_2}
Folgend ist die mathematische Herleitung der Backpropagation für das in Aufgabe 1 gegebene Modell aufgeführt.\\\\
y: Beobachtete Werte der Stichprobe\\
$a_{2}=\hat{\pi}$\\
$\sigma$ = Sigmoid-Funktion
\begin{align*}
    dW_{2} &= \frac{\partial E^{n}}{\partial W_{2}}\\
    &= \frac{\partial E^{n}}{\partial z_{2}}\cdot \frac{\partial z_{2}}{\partial W_{2}}\\
    &= \frac{\partial E^{n}}{\partial a_{2}} \cdot \frac{\partial a_{2}}{\partial z_{2}} \cdot \frac{\partial z_{2}}{\partial W_{2}}\\
    &= \frac{\partial }{\partial a_{2}} \frac{1}{2}(y-a_{2})^{2} \cdot \frac{\partial }{\partial z_{2}} \sigma(a_{1}W_{2} + b_{2})\cdot \frac{\partial }{\partial W_{2}}(a_{1}W_{2}+b_{2})\\
    &= -(y-a_{2}) \cdot \sigma(a_{1}W_{2}+b_{2})(1-\sigma(a_{1}W_{2}+b_{2}))\cdot a_{1}\\
    &= \left[ -\begin{pmatrix}
        y_{1}  -  a_{2;11} \\
        \vdots \\
        y_{n}  -  a_{2;n1}  \\
        \end{pmatrix} \cdot \begin{pmatrix}
                            \sigma (z_{2;11}) \\
                            \vdots \\
                            \sigma (z_{2;n1})  \\
                            \end{pmatrix} \cdot \begin{pmatrix}
                                                    1-\sigma (z_{2;11}) \\
                                                    \vdots \\
                                                    1-\sigma (z_{2;n1})  \\
                                                    \end{pmatrix}\right]^{T} \cdot \begin{pmatrix}
                                                                        a_{1;11} & \cdots & a_{1;1m} \\
                                                                        \vdots & \ddots & \vdots \\
                                                                        a_{1;n1}&\cdots & a_{1;nm}  \\
                                                                        \end{pmatrix}\\\\
    \Rightarrow \delta_{2} &= \frac{\partial E^{n}}{\partial z_{2}}\\
     \delta_{2} &= -(y-a_{2}) \cdot \sigma(a_{1}W_{2}+b_{2})(1-\sigma(a_{1}W_{2}+b_{2}))\\\\
    db_{2} &= \frac{\partial E^{n}}{\partial b_{2}}\\
    &= \frac{\partial E^{n}}{\partial a_{2}} \cdot \frac{\partial a_{2}}{\partial z_{2}} \cdot \frac{\partial z_{2}}{\partial b_{2}}\\
    &= -(y-a_{2}) \cdot \sigma(a_{1}W_{2}+b_{2})(1-\sigma(a_{1}W_{2}+b_{2})) \cdot 1\\
    &=\left[-\begin{pmatrix}
        y_{1}  -  a_{2;11} \\
        \vdots \\
        y_{n}  -  a_{2;n1}  \\
        \end{pmatrix} \cdot \begin{pmatrix}
                            \sigma (z_{2;11}) \\
                            \vdots \\
                            \sigma (z_{2;n1})  \\
                            \end{pmatrix} \cdot \begin{pmatrix}
                                                    1-\sigma (z_{2;11}) \\
                                                    \vdots \\
                                                    1-\sigma (z_{2;n1})  \\
                                                    \end{pmatrix}\right]^{T} \cdot \begin{pmatrix}
                                                                            1 \\
                                                                            \vdots \\
                                                                            1  \\
                                                                            \end{pmatrix} \\\\
    dW_{1} &= \frac{\partial E^{n}}{\partial W_{1}}\\
    &= \delta_{1} \cdot X\\\\
\end{align*}
\begin{align*}    
    \Rightarrow \delta_{1} &= \frac{E^{n}}{\partial z_{1}}\\
    &= \sum_{k} \frac{\partial E^{n}}{\partial z_{2}}\cdot \frac{\partial z_{2}}{\partial z_{1}}\\
    &= \sum_{k} \delta_{2} \cdot \frac{\partial z_{2}}{\partial z_{1}}\\
    &= \sum_{k} \delta_{2}\cdot \frac{\partial }{\partial a_{1}}(a_{1}W_{2}+b_{2})\cdot \frac{\partial }{\partial z_{1}}\sigma(X\cdot W_{1} +b_{1})\\
    &= \sum_{k} \delta_{2}\cdot W_{2} \cdot \sigma(X\cdot W_{1}+b_{1})(1- \sigma(X\cdot W_{1}+b_{1}))\\
    \Rightarrow dW_{1}&= \delta_{2} \cdot \begin{pmatrix}
                        w_{2;11} \\
                        \vdots \\
                        w_{2;m1} \\
                        \end{pmatrix} \cdot \begin{pmatrix}
                                            \sigma (z_{1;11})& \cdots & \sigma(z_{1;1m}) \\
                                            \vdots & \ddots &\vdots\\
                                            \sigma (z_{1;n1}) & \cdots & \sigma (z_{1;nm})  \\
                                            \end{pmatrix} \cdot \begin{pmatrix}
                                                                1-\sigma (z_{2;11}) \\
                                                                \vdots \\
                                                                1-\sigma (z_{2;n1})  \\
                                                                \end{pmatrix} \cdot \begin{pmatrix}
                                                                                         x_{11} & x_{12}\\
                                                                                         \vdots & \vdots\\
                                                                                         x_{n1} & x_{n2}\\
                                                                                        \end{pmatrix} \\\\
    db_{1} &= \delta_{1} \cdot \frac{\partial}{\partial b_{1}}\sigma(X\cdot W_{1}+b_{1})\\
    &= \delta_{1} \cdot 1\\
    &= \delta_{2} \cdot \begin{pmatrix}
                        w_{2;11} \\
                        \vdots \\
                        w_{2;m1} \\
                        \end{pmatrix} \cdot \begin{pmatrix}
                                            \sigma (z_{1;11})& \cdots & \sigma(z_{1;1m}) \\
                                            \vdots & \ddots &\vdots\\
                                            \sigma (z_{1;n1}) & \cdots & \sigma (z_{1;nm})  \\
                                            \end{pmatrix} \cdot \begin{pmatrix}
                                                                1-\sigma (z_{2;11}) \\
                                                                \vdots \\
                                                                1-\sigma (z_{2;n1})  \\
                                                                \end{pmatrix} \cdot 1
\end{align*}
\subsection{Quellcode zu Booklet Teil 2} \label{app:quellcode_booklet_2}
\pagebreak
\subsection{Quellcode zu Booklet Teil 3}
\subsection{Ergänzende Tabellen zu Teil 3}

\begin{table}[h]
	\centering
	\begin{tabular}{ll}
		\hline
		Hyperparameter              & Parameterraum                              \\ \hline
		n\_estimators               & {[}10, 50, 100{]}                          \\
		criterion                   & {[}gini, entropy{]}                    \\
		max\_depth                  & {[}None, 3,5,10, 15, 20, 25, 50{]}         \\
		min\_samples\_split         & {[}2, 5,10, 20, 30, 40{]}                  \\
		min\_samples\_leaf          & {[}1, 2, 5, 10, 20, 40, 100, 200{]}                  \\
		min\_weight\_fraction\_leaf & {[}0, 0.2, 0.4, 0.5{]}                     \\
		max\_features               & {[}None, 5, 10, 15 ,20, 25{]}              \\
		max\_leaf\_nodes            & {[}None, 2 ,10, 100, 150, 200, 300, 500{]} \\
		min\_impurity\_decrease     & {[}0.0, 0.001, 0.002, 0.01, 0.1{]}     \\
		min\_impurity\_split        & {[}0.0, 0.1, 0.2, 0.5, 0.8, 1, 2, 3{]}     \\ \hline
	\end{tabular}
	\caption{\label{table:parameter_grid_univariat} Parameterräume der univariaten \emph{Grid Search}}
\end{table}

\begin{table}[h]
		\centering
\begin{tabular}{lll}
	\hline
	Hyperparameter          & Parameterraum                       & Gewählter Parameter \\ \hline
	max\_depth              & {[}None, 5, 10, 15{]}               & None                \\
	min\_samples\_split     & {[}2, 5, 10{]}                      & 5                   \\
	max\_features           & {[}None, 5, 20, 25{]}               & None                \\
	min\_impurity\_decrease & {[}0.00005, 0.0001, 0.001, 0.003{]} & 0.00005             \\ 
	n\_estimators &  & 100 \\ \hline
\end{tabular}
	\caption{\label{table:parameter_grid_multivariat_forest} Parameterräume und Ergebnisse der multivariaten \emph{Grid Search} des Random Forest}
\end{table}


\begin{table}[]
	\centering
	\begin{tabular}{lll}
	\hline
	Hyperparameter          & Parameterraum                       & Gewählter Parameter \\ \hline
	max\_depth              & {[}None, 5, 10, 15{]}               & None                \\
	min\_samples\_split     & {[}2, 5, 10{]}                      & 5                   \\
	max\_features           & {[}None, 5, 20, 25{]}               & None                \\
	min\_impurity\_decrease & {[}0.00005, 0.0001, 0.001, 0.003{]} & 0.00001             \\ \hline
\end{tabular}
	\caption{\label{table:parameter_grid_multivariat_tree} Parameterräume und Ergebnisse der multivariaten \emph{Grid Search} des Entscheidungsbaums}
\end{table}

\begin{table}[]
	\centering
	\begin{tabular}{lll}
	\hline
	Hyperparameter          & Parameterraum                        & Gewählter Parameter \\ \hline
	max\_depth              & {[}None, 5, 10, 20. 25{]}                & 20                   \\
	min\_samples\_split     & {[}2, 5, 10, 20{]}                   & 2                  \\
	max\_features           & {[}None, 5, 20, 25{]}                & None                  \\
	min\_impurity\_decrease & {[}0.0, 0.0001, 0.0005, 0.001{]} & 0.0             \\ 
	n\_estimators &  & 100 \\ \hline
\end{tabular}
	\caption{\label{table:parameter_grid_multivariat_adaboost} Parameterräume und Ergebnisse der multivariaten \emph{Grid Search} des Adaboost-Verfahrens}
\end{table}

\begin{table}[]
	\centering
	\begin{tabular}{lll}
	\hline
	Hyperparameter          & Parameterraum                        & Gewählter Parameter \\ \hline
	max\_depth              & {[}None, 5, 10, 15{]}                & None                \\
	min\_samples\_split     & {[}2, 5, 10, 20{]}                   & 10                  \\
	max\_features           & {[}None, 5, 20, 25{]}                & 5                   \\
	min\_impurity\_decrease & {[}0.00005, 0.0001, 0.0005, 0.001{]} & 0.0001              \\ 
	n\_estimators &  & 100 \\ \hline
\end{tabular}
	\caption{\label{table:parameter_grid_multivariat_bagging} Parameterräume und Ergebnisse der multivariaten \emph{Grid Search} des Bagging-Verfahrens}
\end{table}



\pagebreak
\subsection{Quellcode zu Booklet Teil 4}